% Created 2016-09-17 Sat 13:31
\documentclass[a4paper]{article}
\usepackage[utf8]{inputenc}
\usepackage[T1]{fontenc}
\usepackage{graphicx}
\usepackage{grffile}
\usepackage{longtable}
\usepackage{wrapfig}
\usepackage{rotating}
\usepackage[normalem]{ulem}
\usepackage{amsmath}
\usepackage{textcomp}
\usepackage{amssymb}
\usepackage{capt-of}
\usepackage[hyperref,x11names]{xcolor}
\usepackage[colorlinks=true,urlcolor=SteelBlue4,linkcolor=Firebrick4]{hyperref}
% --------------INTEGRATION CODE JAVA----------
\usepackage{listings}
\lstset{
language=Java,
basicstyle=\normalsize, % ou ça==> basicstyle=\scriptsize,
upquote=true,
aboveskip={1.5\baselineskip},
columns=fullflexible,
showstringspaces=false,
extendedchars=true,
breaklines=true,
showtabs=false,
showspaces=false,
showstringspaces=false,
identifierstyle=\ttfamily,
keywordstyle=\color[rgb]{0,0,1},
commentstyle=\color[rgb]{0.133,0.545,0.133},
stringstyle=\color[rgb]{0.627,0.126,0.941},
}

\author{Willian Ver Valen Paiva \and Alan Guitard}
\date{\today}
\title{Rapport TP6}
\hypersetup{
 pdfauthor={Willian Ver Valen Paiva Guitard Alan},
 pdftitle={Création d'InputFormat personnalisé},
 pdfkeywords={},
 pdfsubject={},
 pdfcreator={Emacs 25.1.50.1 (Org mode 8.3.5)}, 
 pdflang={fr}
 }
\begin{document}

\maketitle

\tableofcontents

\newpage
\section{Introduction}
L’objectif de ce travail consiste à utiliser et maîtriser les méthodes «setup» et
«cleanup» qui sont impliquées	par l’exercice proposé.
La méthode «setup» est celle qui est appelée avant le «mapper» et où on initialise les
objets qui vont être utilisés dans tous les	«mappers».
La méthode «cleanup» est celle qui est appelée après le «mapper».
\section{Exercice 1}
Pour faire cet exercice nous avons initialisé deux variables dans la méthode Setup; une
qui	s’appelle ktown et l’autre : topk dans lesquelles ktown est une TreeMap et topk dont
la valeur est passée en paramètre par la ligne de commandes.

\begin{lstlisting}
    protected void setup(Context context)
    {
      ktown = new TreeMap<Long,String>();
      topk = context.getConfiguration().getInt("topk", 1);
    }
\end{lstlisting}
\paragraph{map}
Dans	la	liste	 ktown,	 nous	entrons	les	valeurs	des	villes	et	populations	grâce	à	la	méthode
add	 où	la	taille	de	la	liste	est	définie	par	la	valeur	de	topk	.	 Lorsque	nous	appelons	cette
méthode,	elle	regarde	si	la	liste	est	déjà	remplie.	Si	 non,	 elle	ajoute	la	nouvelle	valeur	à	la
liste.	 Si	 elle	 est	 déjà	 remplie,	 elle	 recherche	 la	 plus	 petite	 valeur	 existante	 dans	 la	 liste	 et
la remplace	par	la	valeur	que	nous	voulons	ajouter,	si	cette	dernière	est	plus	grande.

\begin{lstlisting}
    private void add(long a,String town)
    {
      if(ktown.size() < topk)
      {
        ktown.put(a,town);
      }
      else
      {
        long first = ktown.firstKey();
        if(first < a)
        {
          ktown.remove(first);
          ktown.put(a, town);
        }
      }
    }
\end{lstlisting}

Puis,	dans la	méthode	 map,	 on	appelle	la	méthode	 add	 avec	toutes	les	villes.
\begin{lstlisting}
    public void map(Object key, Text value, Context context
        ) throws IOException, InterruptedException {

      String[] line = value.toString().split(",");

      if(!line[4].equals("Population"))
      {
        if(!line[4].equals(""))
        {
          String name = line[2];
          this.add(Long.parseLong(line[4]), name);
        }
      }
    }
\end{lstlisting}
Ensuite, dans	 la	 méthode	 cleanup,	 on	 écrit	 toutes	 les	 valeurs	 de	 la	 liste	 ktown	dans	 le
contexte.

\begin{lstlisting}
    protected  void cleanup(Context context) throws IOException, InterruptedException
    {
      for(long d : ktown.keySet())
      {
        context.write(new Text(ktown.get(d)), new LongWritable(d));
      }
    }
\end{lstlisting}

\paragraph{reducer}
Les	 méthodes	 setup	 et	 cleaner	 	 du	 reducer	 sont	 exactement	 les	 mêmes	 que	 celles	 du
mapper.	Et	dans	le	 reducer	 on	appelle	 add	 pour	toutes	les	valeurs	reçues.

\begin{lstlisting}
    public void reduce(Text key, Iterable<LongWritable> values,
        Context context
        ) throws IOException, InterruptedException {
      for(LongWritable v : values)
      {
        this.add(v.get(),key.toString());
      }
    }
\end{lstlisting}

\section{Exercice 2}
Il s’agit là d’ajouter la classe reducer en tant que combiner au job.
\begin{lstlisting}
    job.setCombinerClass(TP3Reducer.class);
\end{lstlisting}



\section{Exercice 3}
Les tests effectués ont tous été passés avec succès.


\end{document}